\documentclass[a4paper,11pt]{article}
\usepackage[authoryear,round]{natbib}
\usepackage{xr}
\externaldocument[U-]{UsingSchwartz97}
\bibliographystyle{plainnat}
\usepackage{url}
\usepackage{a4wide}
\usepackage{amsmath}
\usepackage{amssymb}
\usepackage[
colorlinks=true,
linkcolor=black,
anchorcolor=black,
citecolor=black,
filecolor=black,
pdfstartview={Fit},
pdfstartpage={1}
]
{hyperref}
\newcommand{\RR}{\textsf{R}}
\newcommand{\vnorm}[1]{\Vert#1\Vert} % Notation: 2-Norm of a Vector
\newcommand{\condexp}[2]{
  \mathbb{\tilde{E}}\left[#1\vert\mathcal{F}_{#2}\right]}
\newcommand{\condexprf}[2]{
  \mathbb{\tilde{E}}\left[#1\vert\mathcal{F}_{#2}\right]}
% Conditional Expectation
\newcommand{\Rpackage}{\textit{schwartz97}}
\newcommand{\sigmaS}{\sigma_S}
\newcommand{\sigmaE}{\sigma_\epsilon}
%%\VignetteIndexEntry{Technical Document}
\title{Schwartz 1997 Two-Factor Model\\Technical Document}
\author{Philipp Erb \and David L\"{u}thi \and Simon Otziger}
\date{\today}
\usepackage{Sweave}
\begin{document}
\maketitle
\begin{abstract}
  The purpose of this document is to give the formulas and relations
  needed to understand the Schwartz two-factor commodity model
  \citep{Schwartz1997}. This includes parameter estimation using the
  Kalman filter, pricing of European options as well as computation of
  risk measures.
\end{abstract}

%\tableofcontents
% <============================================================>
\section{Introduction}\label{sec:intro}
% <============================================================>
This document describes the Schwartz two-factor model to the extent
which is necessary to understand the \RR{} package \Rpackage.  The two
factors are the spot price of a commodity together with its
instantaneous convenience yield. It was introduced in
\cite{Gibson1990} and extended in \cite{Schwartz1997} for the pricing
of futures contracts. \cite{Miltersen1998} and \cite{Hilliard1998}
presented equations for arbitrage free prices of European options on
commodity futures. In what follows we fully rely on the above
mentioned articles and state the corresponding formulas. In addition
we derive the transition density of the two state variables.
% <============================================================>
\section{Model}\label{sec:model}
% <============================================================>
The spot price of the commodity and the instantaneous convenience
yield are assumed to follow the joint stochastic process:

\begin{eqnarray}
  dS_t &=& (\mu - \delta_t) S_t dt + \sigmaS S_t
  dW_S   \label{eq:Pdynamics-S} \\
  d\delta_t &=& \kappa (\alpha - \delta_t) dt + \sigmaE
  dW_\epsilon, \label{eq:Pdynamics-delta}
\end{eqnarray}
with Brownian motions $W_S$ and $W_\epsilon$ under the objective
measure $\mathbb{P}$ and correlation $dW_SdW_\epsilon = \rho dt$.

Under the pricing measure $\mathbb{Q}$ the dynamics are of the form
\begin{eqnarray}
  dS_t &=& (r - \delta_t) S_t dt + \sigmaS S_t
  d\widetilde{W}_S   \label{eq:Qdynamics-S} \\
  d\delta_t &=& [\kappa (\alpha - \delta_t) - \lambda] dt + \sigmaE
  d\widetilde{W}_\epsilon, \label{eq:Qdynamics-delta}.
\end{eqnarray}
where the constant $\lambda$ denotes the market price of convenience
yield risk and $\widetilde{W}_S$ and $\widetilde{W}_\epsilon$ are
$\mathbb{Q}$-Brownian motions. It may be handy to introduce a new
mean-level for the convenience yield process under $\mathbb{Q}$
\begin{equation}
  \label{eq:Q-drift}
 \tilde{\alpha} = \alpha - \lambda / \kappa,
\end{equation}
which leads to the dynamics
\begin{equation}
d\delta_t = \kappa (\tilde{\alpha} - \delta_t) dt + \sigmaE
  d\widetilde{W}_\epsilon.
\end{equation}
% <============================================================>
\section{Distributions}\label{sec:distributions}
% <============================================================>
\subsection{Joint Distribution of State Variables}
% <============================================================>
The log-spot $X_t = \log(S_t)$ and the convenience yield $\delta_t$
are jointly normally distributed. The transition density is
\footnote{A derivation can be found in appendix
  \ref{sec:deriv-joint-distr}.}
\begin{equation}
\label{jointDist}
\begin{pmatrix}X_t \\ \delta_t \end{pmatrix} \sim
\mathcal{N}\left( \begin{pmatrix}\mu_X(t) \\
    \mu_\delta(t) \end{pmatrix},
  \begin{pmatrix}\sigma_X^2(t) & \sigma_{X\delta}(t)  \\ \sigma_{X\delta}(t) & \sigma_\delta^2(t)  \\
    \end{pmatrix} \right),
\end{equation}
with parameters
\begin{align}
  \mu_X(t) &= X_0 + \left(\mu - \frac{1}{2}\sigmaS^2 - \alpha\right) t
  + \left(\alpha  - \delta_0\right) \frac{1 - e^{-\kappa t}}{\kappa}\label{eq:spot-mu}\\
  \mu_\delta(t) &= e^{-\kappa t}\delta_0 + \alpha \left(1 - e^{-\kappa
      t}\right)\label{eq:cy-mu}\\
  \sigma_X^2(t) &= \frac{\sigmaE^2}{\kappa^2}
  \left(\frac{1}{2\kappa}\left(1 - e^{-2\kappa t}\right) -
    \frac{2}{\kappa}\left(1 - e^{-\kappa t}\right) + t\right) +
  2\frac{\sigmaS\sigmaE\rho}{\kappa} \left(\frac{1 - e^{-\kappa
        t}}{\kappa} - t\right) + \sigma^2_St \label{eq:spot-var}\\
  \sigma_\delta^2(t) &= \frac{\sigmaE^2}{2\kappa}\left(1 -
    e^{-2\kappa t}\right) \label{eq:cy-var}\\
  \sigma_{X\delta}(t) &= \frac{1}{\kappa}\left\{\left(\sigmaS\sigmaE\rho -
      \frac{\sigmaE^2}{\kappa}\right)\left(1 - e^{-\kappa t}\right) +
    \frac{\sigmaE^2}{2\kappa}\left(1 - e^{-2\kappa
        t}\right)\right\}\label{eq:spot-cov}.
\end{align}
The mean-parameters given in (\ref{eq:spot-mu}) and
(\ref{eq:cy-mu}) refer to the $\mathbb{P}$-dynamics. To obtain the
parameters under $\mathbb{Q}$ one can simply replace $\mu$ by $r$ and
$\alpha$ by $\tilde{\alpha}$ defined in equation \ref{eq:Q-drift}. Let
the $\mathbb{Q}$-parameters be denoted by $\tilde{\mu}_X(t)$ and
$\tilde{\mu}_{\delta}(t)$.
% <============================================================>
\section{Futures Price}\label{sec:futures-price}
% <============================================================> In
It is worth to mention that the futures and forward price coincide
since in our model the interest rate is assumed to be constant. In the
rest of this document, the statements made about futures contracts
therefore also hold for forward contracts.

Let the futures price at time $t$ with time to maturity $\tau = T - t$
be $G(S_t, \delta_t, t, T)$. For notational convenience we assume $t =
0$ in what follows. At time zero the futures price is given by the
$\mathbb{Q}$-expectation of $S_T$.
\begin{eqnarray}
  \label{eq:futuresPrice}
  G(S_0, \delta_0, 0, T) = \mathbb{E}^{\mathbb{Q}}(S_T) &=& \exp\left\{\tilde{\mu}_X(T) +
    \frac{1}{2}\sigma_X^2(T)\right\}\\
  &=& S_0 e^{A(T) + B(T)\delta_0}
\end{eqnarray}
with
\begin{align}
  A(T) &= \left(r - \tilde{\alpha} + \frac{1}{2}
    \frac{\sigmaE^2}{\kappa^2} - \frac{\sigmaS \sigmaE
      \rho}{\kappa}\right) T  + \frac{1}{4} \sigmaE^2 \frac{1 - e^{-
      2 \kappa T}}{\kappa^3} + \left( \kappa\tilde{\alpha} +
    \sigmaS \sigmaE \rho - \frac{\sigmaE^2}{\kappa} \right) \frac{1 -
    e^{-\kappa T}}{\kappa^2}\label{eq:A-futures},\\
  B(T) &= - \frac{1 - e^{-\kappa T}}{\kappa}\label{eq:B-futures}.
\end{align}
% <============================================================>
\subsection{Distribution of Futures Prices}
% <============================================================>
According to (\ref{eq:futuresPrice}) the futures price follows a
log-normal law. That is, at time zero the $T$-futures price at time
$t$ has the following distribution under $\mathbb{Q}$\footnote{Note
  that the $\mathbb{Q}$-dynamics is primarily interesting to value
  derivatives on the futures price. For simulation studies and dynamic
  financial analysis the real-world ($\mathbb{P}$) dynamics is of
  relevance. To get the $\mathbb{P}$-dynamics all ``tilde-parameters''
  have to be replaced by the ones without tilde.}:
\begin{align}
  \log{G(S_t, \delta_t, t,T)} &\sim \mathcal{N}\left(\mu_G(t, T),
    \sigma_G^2(t, T)\right),\label{eq:distrG}
\end{align}
where
\begin{align}
  \mu_G(t, T) &= \tilde{\mu}_X(t) + A(T-t) + B(T-t)\tilde{\mu}_\delta(t) \label{eq:muG} \\
  \sigma_G^2(t, T) &= \sigma_X^2(t) + 2 B(T-t)\sigma_{X\delta}(t) +
  B(T-t)^2 \sigma_\delta(t). \label{eq:varG}
\end{align}
% <============================================================>
\section{European Commodity Options}
\label{sec:Options}
% <============================================================>
The fair price of a European call option on a commodity futures
contract was derived as a special case of more general models in
\cite{Miltersen1998} and \cite{Hilliard1998}.

Here we give the formula for the two-factor model. In this setting,
the price of a European call option $C$ at time zero with maturity $t$,
exercise price $K$ written on a commodity futures contract with
maturity $T$ is given by
\begin{align}
  C^G &= \mathbb{E}^{\mathbb{Q}}\left[e^{-r\,t}(G(S_t, \delta_t, t, T)
    - K)^+\right]
\end{align}
Since the futures price $G(S_t, \delta_t, t, T)$ is log-normally
distributed we obtain a Black-Scholes type formula for the call price
$C^G$.
\begin{equation}
  \label{callF}
  C^G = P(0,t) \left\{G(0,T) \Phi(d_+) - K \Phi(d_-)\right\}
\end{equation}
with
\begin{align*}
d_\pm &= \frac{\log\frac{G(0,T)}{K} \pm \frac{1}{2}\sigma^2}{\sigma}\\
\sigma^2 &= \sigma_{S}^2 t + \frac{2 \sigma_{S} \sigma_{\epsilon} \rho}{\kappa}
\left(\frac{1}{\kappa} e^{-\kappa T}
  \left(e^{\kappa t} - 1\right) - t\right)\\
&+ \frac{\sigma_{\epsilon}^2}{\kappa^2}
\left(t + \frac{1}{2 \kappa} e^{-2 \kappa T}
  \left(e^{2 \kappa t} - 1\right) - \frac{2}{\kappa}
  e^{-\kappa T}  \left(e^{\kappa t} - 1\right)\right)\notag
\end{align*}
and $\Phi$ being the standard Gaussian distribution
function.

The following put-call parity is established
\begin{equation}
C^G - P^G = P(0,t)\left\{G(0,t) - K\right\}.
\end{equation}
Thus, the price for a European put option $P^G$ at time zero with
maturity $t$, exercise price $K$ written on a commodity futures
contract with maturity $T$ becomes
\begin{equation}
  \label{putF}
  P^G = P(0,t) \left\{K \Phi(-d_-) - G(0,T) \Phi(-d_+)\right\}.
\end{equation}

Remark: For the special case when the exercise time $t$ of the option
and the maturity $T$ of the futures contract coincide, formulas
(\ref{callF}) and (\ref{putF}) still hold. However, the options we
price for $t=T$ are no longer options written on futures contracts
with maturity $T$ but rather options with exercise time $T$, written on
a commodity \emph{spot} contract.


% <============================================================>
\section{Parameter Estimation}\label{sec:Estimation}
% <============================================================>
This section demonstrates an elegant way of estimating the Schwartz
two-factor model. That is estimating the model parameters using the
Kalman filter as in \cite{Schwartz1997}. Subsection
\ref{sec:stateSpace} shows how the Schwartz two-factor model can be
expressed in state space form. Once the model has been cast in this
form the likelihood can be computed and numerically maximized.
% <============================================================>
\subsection{State Space Representation}
\label{sec:stateSpace}
% <============================================================>
Let $\mathbf{y}_t$ denote a $(n\times{1})$ vector of futures prices
observed at date $t$ and $\boldsymbol{\alpha}_t$ denote the
$(2\times{1})$ \emph{state vector} of the spot price and the
convenience yield. The state space representation of the dynamics of
$\mathbf{y}$ is given by the linear system of equations
\begin{eqnarray}
  \mathbf{y}_t &=& \mathbf{c}_t + Z_t \boldsymbol{\alpha}_t + G_t
  \boldsymbol{\eta}_t  \label{eq:measurement}\\
  \boldsymbol{\alpha}_{t + 1} &=& \mathbf{d}_t + T_t \boldsymbol{\alpha}_t +
  H_t \boldsymbol{\epsilon}_t, \label{eq:transition}
\end{eqnarray}
where $\boldsymbol{\epsilon}_t \sim \mathcal{N}(\boldsymbol{0}, I_2)$
and $\boldsymbol{\eta}_t \sim \mathcal{N}(\boldsymbol{0}, I_n).$ $G_t$
and $H_t$ are assumed to be time-invariant.  The errors
(``innovations'') in the measurement equation (\ref{eq:measurement})
are further assumed to be independent in the implementation of this
package
\begin{equation}
  \label{eq:1}
  \mathbf{G}_t\mathbf{G}^{'}_t = \begin{pmatrix}g_{11}^2&&\\
    &\ddots&\\ &&g_{nn}^2\end{pmatrix}.
\end{equation}
Using the functions $A(\cdot)$ and $B(\cdot)$ defined in
(\ref{eq:A-futures}) and (\ref{eq:B-futures}) the components of
the state space representation (\ref{eq:measurement}) and
(\ref{eq:transition}) are
\begin{eqnarray}
  \boldsymbol{\alpha}_{t+\Delta t}&=&\begin{pmatrix}X_{t+\Delta t}\\\delta_{t+\Delta t}\end{pmatrix}\\
  \mathbf{T}_t &=&
  \begin{pmatrix}
    1 & \frac{1}{\kappa}\left(e^{-\kappa
        \Delta t} - 1\right)\\
    0 & e^{-\kappa \Delta t}
  \end{pmatrix}\\
  \mathbf{d}_t &=&
  \begin{pmatrix}
    \left(\mu - \frac{1}{2} \sigmaS^2 - \alpha \right) \Delta t + \frac{\alpha}{\kappa} \left(1 - e^{-\kappa \Delta t}\right)\\
    \alpha \left(1 - e^{-\kappa \Delta t}\right)
  \end{pmatrix}\\
  \mathbf{H}_t\mathbf{H}^{'}_t &=&
  \begin{pmatrix}
    \sigma_X^2(\Delta{t}) & \sigma_{X \delta}(\Delta{t}) \\
    \sigma_{X \delta}(\Delta{t}) & \sigma_\delta^2(\Delta{t})
  \end{pmatrix}
\end{eqnarray}

\begin{eqnarray}
  \mathbf{y}_t =
  \begin{pmatrix}
    \log{G_t(1)}\\
    \vdots\\
    \log{G_t(n)}
  \end{pmatrix}
  & \mathbf{Z}_t =
  \begin{pmatrix}
    1 & B(m_t(1))\\
    \vdots & \vdots\\
    1 & B(m_t(n))
  \end{pmatrix}\\
  \mathbf{c}_t =
  \begin{pmatrix}
    A(m_t(1))\\
    \vdots \\
    A(m_t(n))
  \end{pmatrix}
  & \mathbf{G}_t\mathbf{G}^{'}_t =
  \begin{pmatrix}
    g_{11}^2&&\\
    &\ddots&\\
    &&g_{nn}^2
  \end{pmatrix}
\end{eqnarray}

where $X_t=\log{S_t}$, $\Delta{t}=t_{k+1}-t_k$ and $m_t(i)$ denotes
the remaining time to maturity of the $i$-th closest to
maturity futures $G_t(i)$.
% <============================================================>
\clearpage
% <============================================================>
\begin{appendix}
% <============================================================>
\section{Derivation of the joint distribution}\label{sec:deriv-joint-distr}
% <============================================================>
The joint dynamics of the commodity log-price $X_t = \log{S_t}$ and
the spot convenience yield $\delta_t$ reads in an orthogonal
decomposition of (\ref{eq:Pdynamics-S}) and (\ref{eq:Pdynamics-delta})

\begin{eqnarray}
  dX_t &=& \left(\mu - \delta_t - \frac{1}{2}\sigmaS^2\right)dt +
  \sigmaS \sqrt{1 - \rho^2 }dW_t^1 + \sigmaS \rho dW_t^2 \label{eq:Pdynamics-X-ortho}\\
  d\delta_t &=& \kappa (\alpha - \delta_t) dt + \sigmaE dW_t^2, \label{eq:Pdynamics-delta-ortho}
\end{eqnarray}
Equation (\ref{eq:Pdynamics-delta-ortho}) can be solved using the
substitution $\tilde{\delta}_t = e^{\kappa t}\delta_t$ and Ito's lemma.
\begin{equation}
\label{eq:delta-integrated}
\begin{split}
  \delta_t = e^{-\kappa t}\delta_0 + \alpha \left(1 -
    e^{-\kappa t}\right) +
  \sigmaE e^{-\kappa t} \int_0^te^{\kappa u}dW_u^2
\end{split}
\end{equation}
Plugging (\ref{eq:delta-integrated}) into (\ref{eq:Pdynamics-X-ortho})
gives
\begin{eqnarray}
  \label{eq:x}
  X_t &=& X_0 + \int_0^t dX_u\\
  &=& X_0 + \left(\mu - \frac{1}{2}\sigmaS^2\right) t -  \int_0^t
  \delta_u du  + \int_0^t  \sigmaS \sqrt{1 - \rho^2} dW_u^1 + \int_0^t  \sigmaS \rho dW_u^2.
\end{eqnarray}
Let's have a look at the integral $\int_0^t \delta_u du$.
\begin{equation}\label{eq:integra-deltat}
\begin{split}
  \int_0^t \delta_u du = \int_0^t e^{-\kappa u}\delta_0 du + \int_0^t \alpha \left(1 -
    e^{-\kappa u}\right) du + \int_0^t\sigmaE e^{-\kappa u} \left(\int_0^u
  e^{\kappa s}dW_s^2 \right) du
\end{split}
\end{equation}
For the integral $\int_0^te^{-\kappa u}\left(\int_0^u e^{\kappa
    s}dW_s^2\right) du$ we use Fubini's theorem to interchange the
order of integration:
\begin{eqnarray}
  \int_0^t\left(\int_0^u  e^{-\kappa u} e^{\kappa
      s}dW_s^2\right) du &=& \int_0^t\left(\int_s^t  e^{-\kappa u} e^{\kappa
      s}du\right) dW_s^2\\
  &=& \int_0^t \frac{1}{\kappa}\left(1 - e^{-\kappa (t - s)}\right)
  dW_s^2 \label{eq:fubini}
\end{eqnarray}
Plugging eq. (\ref{eq:fubini}) into eq. (\ref{eq:integra-deltat}) and
solving the Riemann integrals yields
\begin{equation}\label{eq:integra-deltat-fubini}
\begin{split}
  \int_0^t \delta_u du =  \frac{\delta_0}{\kappa} \left(1 -
    e^{-\kappa t}\right) + \alpha t - \frac{\alpha}{\kappa} \left(1 - e^{-\kappa
      t}\right) + \sigmaE \int_0^t \frac{1}{\kappa}\left(1 -
    e^{-\kappa (t - s)}\right) dW_s^2.
\end{split}
\end{equation}
This leaves us with the following expression for $X_t$:
\begin{eqnarray}
  \label{eq:x}
  X_t &=& X_0 + \left(\mu - \frac{1}{2}\sigmaS^2 - \alpha\right) t +
  (\alpha - \delta_0) \frac{1 - e^{-\kappa t}}{\kappa} +  \nonumber \\
  && +  \int_0^t \sigmaS \sqrt{1 - \rho^2} dW_u^1
  + \int_0^t \left\{\sigmaS \rho + \frac{\sigmaE}{\kappa}\left(e^{-\kappa (t - u)} - 1\right)\right\} dW_u^2.
\end{eqnarray}
The log-spot $X_t$ and the convenience yield $\delta_t$ are jointly
normally distributed with expectations
\begin{eqnarray}
  \label{eq:5}
  E(X_t) &=& \mu_X = X_0 + \left(\mu - \frac{1}{2}\sigmaS^2 - \alpha\right) t
  + \left(\alpha  - \delta_0\right) \frac{1 - e^{-\kappa t}}{\kappa}\\
  E(\delta_t) &=& \mu_\delta = e^{-\kappa t}\delta_0 + \alpha \left(1 - e^{-\kappa
      t}\right).
\end{eqnarray}
The variance are obtained using expectation rules for Ito integrals
and the Ito isometry.

\begin{align}
  var(X_t) &= \sigma_X^2 = \frac{\sigmaE^2}{\kappa^2}
  \left\{\frac{1}{2\kappa}\left(1 - e^{-2\kappa t}\right) -
    \frac{2}{\kappa}\left(1 - e^{-\kappa t}\right) + t\right\} +
  2\frac{\sigmaS\sigmaE\rho}{\kappa} \left(\frac{1 - e^{-\kappa
        t}}{\kappa} - t\right) + \sigma^2_St \\
  var(\delta_t) &= \sigma_\delta^2 = \frac{\sigmaE^2}{2\kappa}\left(1 -
    e^{-2\kappa t}\right) \\
  cov(X_t, \delta_t) &= \sigma_{X\delta} = \frac{1}{\kappa}\left[\left(\sigmaS\sigmaE\rho -
      \frac{\sigmaE^2}{\kappa}\right)\left(1 - e^{-\kappa t}\right) +
    \frac{\sigmaE^2}{2\kappa}\left(1 - e^{-2\kappa t}\right)\right]
\end{align}
% The covariance between $X_t$ and $\delta_t$ is computed as follows
% \begin{align}
% \sigma_{X\delta} &= \mathbb{E}\left[X_t\delta_t\right] -
% \mathbb{E}X_t\mathbb{E}\delta_t \\
% &= \mathbb{E}\left[
% \int_0^t\left[\sigmaS +
%   \frac{\sigmaE\rho}{\kappa}
% \left(e^{-\kappa(t - u)} - 1\right)\right]dW_u^1
% \times
% \sigmaE\rho^{-\kappa
%   t}\int_0^te^{\kappa u}dW_u^1
% \right] \notag \\
% &+ \mathbb{E}\left[
% \frac{\sigmaE\sqrt{1 -
%   \rho^2}}{\kappa}\int_0^t\left(e^{-\kappa(t
%   - u)} - 1\right)dW_u^2
% \times
% \sigmaE\sqrt{1 -
%   \rho^2}e^{-\kappa t}\int_0^te^{\kappa
%   u}dW_u^2
% \right] \\
% &=
% \int_0^t\left\{\sigmaS\sigmaE\rho^{-\kappa(t
%     - u)} +
%   \frac{\sigmaE^2\rho^2}{\kappa}\left(
%   e^{-2\kappa(t - u)} - e^{-\kappa(t - u)}
% \right)\right\}du
% \notag \\
% &+ \int_0^t \frac{\sigmaE^2\left(1 -
%     \rho^2\right)}{\kappa}\left(
%   e^{-2\kappa(t - u)} - e^{-\kappa(t - u)} \right)du.
% \end{align}
\end{appendix}
\bibliography{SpotCommodity}
\end{document}
